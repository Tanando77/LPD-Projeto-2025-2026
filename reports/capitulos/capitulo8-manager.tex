\chapter{Gestão de Segredos: 7-Manager}
\label{cap8}

\section{Fundamentação Teórica: Armazenamento Seguro e TOTP}
O requisito 1.7 foca-se na gestão de passwords. Guardar passwords em texto limpo é uma falha grave. A solução passa por cifrar os dados (URL/User/Pass) e proteger o acesso ao cofre com autenticação de dois fatores (2FA). O TOTP (\textit{Time-based One-Time Password}) gera códigos temporários, garantindo que, mesmo que a password mestre seja roubada, o atacante não aceda aos dados sem o segundo fator.

\section{Implementação do Vault}
O script \texttt{7-manager.py} gere o \texttt{passwords.json}, enquanto o \texttt{7-seguranca.py} lida com a autenticação 2FA.

\section{Análise Detalhada do Código}
A segurança do ficheiro JSON é garantida pela cifragem dos campos sensíveis:

\begin{lstlisting}[caption={Gestão de Cofre Cifrado}]
import json
import pyotp

def verify_2fa(user_token):
    # Implementação de TOTP com pyotp.
    totp = pyotp.TOTP("BASE32SECRETKEY")
    return totp.verify(user_token)

def save_to_json(service, password):
    # Ciframos a password antes de escrever no passwords.json
    encrypted_pw = encrypt_with_master_key(password)
    # Persistência estruturada em JSON
    with open('passwords.json', 'w') as f:
        json.dump({service: encrypted_pw}, f)
\end{lstlisting}

Este capítulo conclui o projeto, abordando a segurança do utilizador final e a proteção de segredos digitais.