\chapter{Mensagens Seguras: 5-Messenger}
\label{cap6}

\section{Fundamentação Teórica: Confidencialidade e Integridade}
A troca de mensagens segura (Requisito 1.5) exige que os dados sejam protegidos contra interceção (\textit{Eavesdropping}). Isto é alcançado através de criptografia. Além disso, o requisito 1.6 pede backups para garantir a disponibilidade. A utilização de criptografia assimétrica garante que apenas o destinatário legítimo possa ler a mensagem, mesmo que esta seja intercetada no servidor.

\section{Implementação Cliente-Servidor}
O sistema utiliza \texttt{5-server.py} para gerir as conexões e o arquivamento, enquanto o \texttt{5-crypto.py} fornece as primitivas de segurança.

\section{Análise Detalhada do Código}

O foco está na proteção dos dados antes do envio pelo socket:

\begin{lstlisting}[caption={Cifragem de Mensagens no 5-crypto.py}]
from cryptography.fernet import Fernet

def encrypt_message(message, key):
    # Utilizamos a biblioteca cryptography para garantir padrões modernos.
    f = Fernet(key)
    # A mensagem é convertida para bytes e cifrada.
    encrypted_data = f.encrypt(message.encode())
    return encrypted_data
\end{lstlisting}

O ficheiro \texttt{backup\_Fernando.bck} serve como prova da implementação do sistema de recuperação de dados, garantindo que o histórico de mensagens multiutilizador está salvaguardado.