\chapter{Introdução}
\label{intro}

Este projeto foi desenvolvido no âmbito da disciplina Linguagens de Programação Dinâmicas (LPD) e tem como principal objetivo aplicar, de forma prática, os conceitos abordados ao longo da unidade curricular. Para esse efeito, foram concebidas e implementadas diversas ferramentas em Python, focadas na programação em redes e na análise de segurança, incluindo um scanner de portas, um mecanismo de comunicação cliente-servidor (chat) e uma simulação de ataques de negação de serviço (DoS), com fins exclusivamente académicos.

O desenvolvimento do projeto decorreu em ambiente Windows, com recurso a uma máquina virtual Kali Linux para apoio em testes e validação de funcionalidades relacionadas com redes e segurança. Através deste trabalho, pretende-se não só consolidar conhecimentos técnicos de programação dinâmica, mas também promover uma melhor compreensão dos riscos, técnicas e mecanismos associados à segurança de sistemas e redes informáticas.