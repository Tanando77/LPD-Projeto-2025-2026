\chapter{Proteção Dinâmica: Port Knocking}
\label{cap7}

\section{Fundamentação Teórica: Segurança por Obscuridade Ativa}
O Port Knocking é um método para abrir portas numa firewall através de uma sequência de tentativas de conexão a portas fechadas. Teoricamente, funciona como uma "combinação" secreta. Enquanto a sequência não for batida corretamente, o porto real (ex: SSH na 22) permanece invisível para scanners (como o do Capítulo 1), protegendo contra vulnerabilidades \textit{zero-day}.

\section{Implementação da Máquina de Estados}
O servidor (\texttt{6-knock-servidor.py}) monitoriza os pacotes e o cliente (\texttt{6-knock-cliente.py}) executa a sequência.

\section{Análise Detalhada do Código}

A lógica reside na validação da sequência temporal:

\begin{lstlisting}[caption={Validação de Sequência no Servidor}]
# Sequência definida em 6-Instructions.txt
knock_sequence = [1000, 2000, 3000]
current_state = 0

def on_packet(received_port):
    global current_state
    if received_port == knock_sequence[current_state]:
        current_state += 1
        if current_state == len(knock_sequence):
            # Se a sequência estiver completa, abre a firewall
            open_ssh_port()
    else:
        current_state = 0 # Reinicia se falhar a sequência
\end{lstlisting}

Este módulo, embora adicional, reforça a robustez da aplicação ao interagir diretamente com as regras de firewall do Linux.