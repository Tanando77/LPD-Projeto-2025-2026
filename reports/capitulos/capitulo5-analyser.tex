\chapter{Análise Forense e Logs: 4-Analyser.py}
\label{cap5}

\section{Fundamentação Teórica: Análise de Logs e SIEM}
O enunciado exige a análise de ficheiros de log (SSH, HTTP) para detetar intrusões. Este processo é vital em Forense Digital. A análise consiste em identificar padrões de ataque (como Brute Force), extrair o carimbo temporal (\textit{timestamp}) e a origem. A integração com GeoIP permite transformar um endereço IP numa localização geográfica, auxiliando na identificação da proveniência dos ataques.

\section{Arquitetura e Persistência}
O módulo é um ecossistema que utiliza \texttt{re} (Regex) para o \textit{parsing}, \texttt{sqlite3} para armazenamento (ficheiro \texttt{projeto\_seguranca.db}) e \texttt{report.py} para o output.

\section{Análise Detalhada do Código}

O script \texttt{4-Analyser.py} foca-se na extração inteligente de dados:

\begin{lstlisting}[caption={Extração de IPs e Registo em Base de Dados}]
import re
import sqlite3

# Expressão regular para isolar o IP de uma linha de log de falha SSH.
pattern = r"Failed password for .* from (\d+\.\d+\.\d+\.\d+)"

def process_logs(log_file):
    with open(log_file, 'r') as f:
        for line in f:
            match = re.search(pattern, line)
            if match:
                ip_address = match.group(1)
                # Inserção no projeto_seguranca.db via db.py
                save_incident(ip_address)
\end{lstlisting}