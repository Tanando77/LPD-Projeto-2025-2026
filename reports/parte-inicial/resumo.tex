\chapter{Resumo}
\section*{\textit{\TITULO}\\  {\small{\textit{\SUBTITULO}}}}


\textit{
Este trabalho apresenta o desenvolvimento de um conjunto de ferramentas em Python, concebidas no âmbito da disciplina Linguagens de Programação Dinâmicas (LPD), como parte da componente prática do curso de Engenharia de Segurança Informática. O projeto, implementado em ambiente Windows com suporte de uma máquina virtual Kali Linux, inclui a criação de utilitários inspirados em ferramentas de rede e segurança, tais como um scanner de portas (semelhante ao Nmap), um atacante de DoS / Syn Flood, e um sistema de chat básico, entre outros módulos programados. O objetivo principal foi aplicar e consolidar conceitos de programação dinâmica em Python, explorando funcionalidades de manipulação de redes e comunicação entre processos, bem como compreender implicações de segurança associadas a essas técnicas. Os resultados demonstram a capacidade de integrar funcionalidades de rede avançadas através de scripts em Python, fornecendo uma base prática para futuros estudos em programação e segurança de sistemas.
\\}





\textbf{Palavras-chave}: \textit{Python, Linguagens de Programação Dinâmicas, Programação em Redes, Segurança Informática, Port Scanning, UDP flood, SYN flood, Ataques de Negação de Serviço (DoS), Comunicação Cliente-Servidor, gestor de passwords}.