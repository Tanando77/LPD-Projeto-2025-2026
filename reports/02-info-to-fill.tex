% Para preencher 

\newcommand{\ESCOLA}{Escola Superior de Tecnologia e Gestão}
%\newcommand{\ESCOLA}{Escola Superior de Educação}
%\newcommand{\ESCOLA}{Escola Superior Agrária}
%\newcommand{\ESCOLA}{Escola Superior de Saúde}


\newcommand{\CURSO}{Mestrado em Engenharia de Segurança Informática}

\newcommand{\TITULO}{Linguagens de Programação Dinâmicas\\Aplicação de Segurança Informática}
\newcommand{\TITULOINGLES}{Dynamic Programming Languages\\Computer Security Application}
\newcommand{\SUBTITULO}{}


\newcommand{\NOMEALUNO}{Fernando Miguel Borges Costa da Silva\\Aluno 28084}

\newcommand{\LOCAL}{Beja}
\newcommand{\DATA}{10 de fevereiro de 2026}


 %se for um estágio deve ser retirado o comentário da linha seguinte e indicar o orientador na entidade de acolhimento do estágio
%\newcommand{\ORIENTADORENTIDADE}{Título académico e nome do(a) orientador(a) na entidade de acolhimento, NomeDaEntidade, por exemplo "Eng. Nome Completo ou Abreviado}

\newcommand{\ORIENTADORIPBEJAA}{Professor Armando Ventura}
% se existir segundo orientador do IPBeja, retirar o comentário da linha seguinte
%\newcommand{\ORIENTADORIPBEJAB}{Colocar o título Académico e nome do(a) segundo(a) docente orientador(a), se existente} 


%Completar e comentar um dos seguintes dois \newcommand
%\newcommand{\DECLARACAOPROJETO}{Relatório de projeto de fim de curso apresentado na\linebreak \ESCOLA{} do Instituto Politécnico de Beja}
%\newcommand{\DECLARACAO}{
%Implementação de Cifras Simétricas em Python
%Relatório de projeto
%Relatório de estágio realizado na/no.................
%}

% retirar comentário e preencher se existente
%\newcommand{\DEDICATORIA}{ texto a colocar}
